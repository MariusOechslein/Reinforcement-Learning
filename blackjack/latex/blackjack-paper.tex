\documentclass[conference]{IEEEtran}
\IEEEoverridecommandlockouts
% The preceding line is only needed to identify funding in the first footnote. If that is unneeded, please comment it out.
\usepackage{cite}
\usepackage{amsmath,amssymb,amsfonts}
\usepackage{algorithmic}
\usepackage{graphicx}
\usepackage{textcomp}
\usepackage{xcolor}
\def\BibTeX{{\rm B\kern-.05em{\sc i\kern-.025em b}\kern-.08em
    T\kern-.1667em\lower.7ex\hbox{E}\kern-.125emX}}
\begin{document}

\title{Reinforcment Learning for Blackjack}

\author{
    \IEEEauthorblockN{Marius Oechslein}
    \IEEEauthorblockA{
        \textit{Faculty of Computer Science and Business Information Systems} \\
        \textit{University of Applied Sciences Würzburg-Schweinfurt}\\
        Würzburg, Germany \\
        marius.oechslein@study.thws.de
    }
}
\maketitle

% Structure:
% 0. Abstract
%	. Zusammenfassung von ganzen Paper (am Ende schreiben!)
% 1. Introduction
%	. Blackjack was solved by Edward Thorb in 1960 and by numerous Reinforcement Learning papers since then (Establishing Research Area)
%	. To my knowledge the does not focus on <Insert my variations> (Establising a Niche)
%	. To investigate this area, I <Insert my approach> (Occupying the Niche)
% 2. Methods
%	. Section: Blackjack (Basic Description of the Game, State spaces (Math), Possible Actions (Math), Which Actions do I permit?)
%	. Section: Reinforcement Learning Methods (Monte Carlo (Math), ...)
%	. Section: My two variations and why they are interesting
% 3. Results
%	. Section: First Variation
%	. Section: Second Variation
% 4. Discussion
% 5. Conclusion



\begin{abstract}
This document is a model and instructions for \LaTeX.
This and the IEEEtran.cls file define the components of your paper [title, text, heads, etc.]. *CRITICAL: Do Not Use Symbols, Special Characters, Footnotes, 
or Math in Paper Title or Abstract.
\end{abstract}

\begin{IEEEkeywords}
	Blackjack, Reinforcement Learning, Monte Carlo
\end{IEEEkeywords}

\section{Introduction}
% TODO
\begin{enumerate}
	\item Blackjack was solved by Edward Thorb in 1960 and by numerous Reinforcement Learning papers since then (Establishing Research Area)
	\item To my knowledge the does not focus on <Insert my variations> (Establising a Niche)
	\item To investigate this area, I <Insert my approach> (Occupying the Niche)
\end{enumerate}

\section{Methods}

\subsection{The game of Blackjack}
% TODO:
\begin{enumerate}
	\item Basic Description of the Game
	\item State Spaces (Math)
	\item Actions (Math) (Doubling Down, Hitting, Splitting, ...)
	\item Which Actions do I permit in my implementation? and why?
\end{enumerate}

\subsection{Reinforcement Learning method - Monte Carlo}
% TODO: 

\subsection{<Insert my variations>}
% TODO: 



\section{Results}


\subsection{<Insert First Variation>}

\subsection{<Insert Second Variation>}


\section{Discussion}


\section{Conclusion}



%\begin{table}[htbp]
%\caption{Table Type Styles}
%\begin{center}
%\begin{tabular}{|c|c|c|c|}
%\hline
%\textbf{Table}&\multicolumn{3}{|c|}{\textbf{Table Column Head}} \\
%\cline{2-4} 
%\textbf{Head} & \textbf{\textit{Table column subhead}}& \textbf{\textit{Subhead}}& \textbf{\textit{Subhead}} \\
%\hline
%copy& More table copy$^{\mathrm{a}}$& &  \\
%\hline
%\multicolumn{4}{l}{$^{\mathrm{a}}$Sample of a Table footnote.}
%\end{tabular}
%\label{tab1}
%\end{center}
%\end{table}

%\begin{figure}[htbp]
%\centerline{\includegraphics{fig1.png}}
%\caption{Example of a figure caption.}
%\label{fig}
%\end{figure}

\begin{thebibliography}{00}
\bibitem{b1} G. Eason, B. Noble, and I. N. Sneddon, ``On certain integrals of Lipschitz-Hankel type involving products of Bessel functions,'' Phil. Trans. Roy. Soc. London, vol. A247, pp. 529--551, April 1955.
\end{thebibliography}

\end{document}
